Persistence has proved to be a valuable tool to analyze real world data robustly.
Often used implicitly, the precise statement of this property and its proof appears here.
We discuss two versions of a conjecture attributed to M. Barr.
These examples were not fabricated to illustrate the abstract possibility of misbehavior.
Rather, they are drawn from the literature.
A number of examples are considered.
We present two other characterizations.
The construction is performed in two steps.
We only treat the branching side.
The symbolic and categorical structures are thereby shown to be equivalent.
The definitions can be read independently.
We give a categorical discussion of such results.
Adamek and Sousa's result follows from ours.
The symmetric case can easily be recovered.
Our result relies heavily on some unpublished work of A. Kock from 1989.
Some properties and sufficient conditions for existence of the construction are examined.
Having many corollaries, this was an extremely useful result.
Moreover, as the authors soon suspected, it specializes a much more general result.
We add some guiding examples.
In this article, we present a general construction of such an extension.
The main theorem comes close to a characterisation of this phenomenon.
We provide various examples of this situation of a combinatorial nature.
For the elementary case, little more is known.
A simple example is computed in explicit detail.
The constructions have the same objects, but are rather different in other ways.
We also point out interesting open problems concerning the Dialectica construction.
We give here a positive answer to this question.
A question was left open: is there more structure yet to be defined?
The mistaken version is used later in that paper.
Some consequences and applications are presented.
In those two cases, the functors in question may have surprisingly opulent structures. 	 		
We show that their use can be avoided and all remaining results remain correct.
See note on p. 24.
Our paper extends these ideas somewhat.
A comparison to other approaches will be made in the introduction.
The present paper starts by supplying this last clause with a precise meaning.
We conclude with applications to examples.
But, is this really the correct level of generalisation?
In this paper we will give a new, elementary proof of this result.
We clarify details of that work.
We also discuss some new examples and results motivated by this characterization.
Further applications are given.
The paper presents algebraic and logical developments. 							
We define and prove the core of what is required.
We also give various examples and counterexamples.
We finish by two questions about the problem suggested by the title of this text. 				
We show that the answer is positive by building some examples.			
