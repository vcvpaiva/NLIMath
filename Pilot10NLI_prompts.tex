%% Building an NLI corpus from TAC sentences

1. We use a number of interesting categories related to probability theory.
E. There exists a number of interesting categories related to probability theory.
C. There are no interesting categories related to probability theory.
N. There are no interesting categories related to topology.

2. A notion of central importance in categorical topology is that of topological functor.
E. Topological functor is a notion of categorical topology.
C. There are no notions of importance in categorical topology.
N. There are many  notions of central importance in categorical topology.

3. In the nilpotent case, this nerve is known to be a Kan complex.
E. There are nerves that are Kan complexes.
C. In the nilpotent case, this nerve is not known to be a Kan complex.
N. This nerve is not known to be a Kan complex.

4. We worked through numerous examples to demonstrate the power of these notions.
E. We worked through two examples to demonstrate the power of these notions.
C. We did not work through numerous examples to demonstrate the power of these notions.
N. We worked through numerous examples to deny the power of these notions.

5. If the category is additive, we define a sheaf of categories of analytic functions.
E. If the category is additive, we define a sheaf of categories of functions.
C. If the category is additive, we do not define a sheaf of categories of analytic functions.
N. we can define a sheaf of categories of analytic functions.

6. We use these relations to define analytic versions of Arakelov compactifications of affine arithmetic varieties.
E. We can define analytic versions of Arakelov compactifications of affine arithmetic varieties.
C. These relations cannot be used to define analytic versions of Arakelov compactifications of affine arithmetic varieties.
N. We use these relations to define analytic versions of  compactifications of affine arithmetic varieties.

7. They are used in the paper only to prove Corollary~8.3.
E. They are used in the paper  to prove Corollary~8.3.
C. They are not used in the paper  to prove Corollary~8.3.
N. They are used in the paper  to prove Corollary~8.5.

8. ***A proof of this corollary is given without them.
E. There exists a proof of this corollary.
C. There is no proof of this corollary without them.
N. A proof of this corollary is given with these

9. Here ``balanced'' can be omitted if the category is additive.
E. For additive categories ``balanced" can be omitted here.
C. Here ``balanced'' cannot be omitted if the category is additive.
N. Here ``balanced'' can be omitted.

10. We introduce the notion of mutation pairs in pseudo-triangulated categories.
E. We introduce the notion of mutation pairs.
C. We cannot introduce the notion of mutation pairs in pseudo-triangulated categories.
N. We introduce the notion of mutation pairs in pseudo-triangulated categories.

11. This result unifies many previous constructions of quotient triangulated categories.
E: This result unifies some previous constructions of quotient triangulated categories.
C: This result is not related to any of the previous constructions of quotient triangulated categories.
N: This result does not subsume some of the previous constructions of quotient triangulated categories.
%%Comment: it is not really clear what exactly "unify" means

12. We study extra assumptions on pretopologies that are needed for this theory.
E: We investigate additional assumptions on pretopologies that are needed for this theory.
C: However, we will not be concerned with extra assumptions on pretopologies needed for this theory.
N: We do not study the basic assumptions on pretopologies that are needed for this theory.

13. We check these extra assumptions in several categories with pretopologies.
E: We check these extra assumptions in at least one category with a pretopology.
C: We will not check these extra assumptions in categories with pretopologies.
N: We check these extra assumptions in many categories with pretopologies.

14. Functors between groupoids may be localised at equivalences in two ways.
E: Functors between groupoids can be localised at equivalences.
C: Unfortunately, it is not possible to localise functors between groupoids at equivalences.
N: Localisation of functors between groupoids is used to prove Theorem 5.3.

15. We show that both approaches give equivalent bicategories.
E: Both approaches yield equivalent bicategories.
C: It was shown that these two approaches give distinct bicategories.
N: Both approaches give the same bicategory.

16. In this paper, we use the language of operads to study open dynamical systems.
E: We study dynamical systems in this paper.
C: We will not be concerned with the language of operads.
N: The study of open dynamical systems requires the language of operads.

17. The syntactic architecture of such interconnections is encoded using the visual language of wiring diagrams.
E: Wiring diagrams are related to the syntactic architecture of such interconnections.
C: Such interconnections lack any syntactic structure.
N: The only way to encode the syntactic structure of such interconnections is by means of the visual language of wiring diagrams.

18.  Moreover it enables us to characterise operads as categorical polynomial monads in a canonical way.
E: Operads can be characterised as categorical polynomial monads.
C: Operads can be characterised as categorical polynomial monads; however, no canonical way of doing so exists.
N: There is exactly one way to characterise operads as categorical polynomial monads.

19. We have two useful gradings related by isomorphisms which change the degree.
E: There exist some gradings related by isomorphisms which change the degree.
C: No two gradings which change the degree are isomorphic.
N: There exist many pairs of gradings related by isomorphisms which change the degree.

20.  The result is a double category C//G which describes the local symmetries of C.
E: The result is a category.
C: The result is a Riemann surface of genus 3 with 2 marked points.
N: The result describes both local and non-local symmetries of C.

21. There are few known computable examples of non-abelian surface holonomy.
E: There are some known examples of non-abelian surface holonomy.
C: There are no known computable examples of non-abelian surface holonomy.
N: There are few known examples of non-abelian surface holonomy.

22. Using these ideas, we also prove that magnetic monopoles form an abelian group.
E:  Using these ideas, we also prove that magnetic monopoles form a group.
C:  Using these ideas, we disprove the conjecture that magnetic monopoles form a group.
N: Using these ideas, we also prove that monopoles form a semigroup.

23.  We introduce a 3-dimensional categorical structure which we call intercategory.
E: We introduce a 3-dimensional categorical structure.
C: In a previous paper, we introduced a 3-dimensional categorical structure which we called intercategory.
N: An intercategory is a category with a 3-dimensional intercategorical structure.

24. We show that these fit together to produce a strict triple category of intercategories.
E: We show that these fit together to produce a category of intercategories.
C: We doubt that these fit together to produce a strict triple category of intercategories.
N: Three intercategories fit together to produce a strict triple.

25. This is the third paper in a series.
E: This paper is part of a series.
C: This is the fourth paper in a series.
N: This is the third paper on this topic.

26. The effect of any bundle of Lie groups is trivial.
E: Lie groups sometimes appear in bundles.
C: The effect of any bundle of Lie groups is 0.
N: Groups always have effects.

27.  All quotients of a given Lie groupoid determine the same effect.
E:  Any two quotients of a given Lie groupoid determine the same effect.
C: Lie groupoids effectively have just one quotient.
N: A Lie groupoid has either zero or one quotient.

28. Our analysis is relevant to the presentation theory of proper smooth stacks.
E: Proper smooth stacks may sometimes be presented.
C: Our analysis does not have anything to say about the presentation theory of proper smooth stacks.
N: Proper smooth stacks may always be presented.

29) This paper extends the Day Reflection Theorem to skew monoidal categories.
E:  This paper extends the Day Reflection Theorem to a family of monoidal categories.
C: This paper derives the Day Reflection Theorem from skew monoidal categories.
N: This paper extends the Day Reflection Theorem to monoidal categories.

30) Let C be a finite category.
SKIPPED

31) We also give a presentation for FinRelk.
E: We also exhibit a presentation for FinRelk.
C: There is no presentation for FinRelk.
N: This is the first time that anyone gives a presentation for FinRelk.

 

